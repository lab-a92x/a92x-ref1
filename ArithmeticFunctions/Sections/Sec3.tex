\subsection{Further prefix sum optimization}
Again we are interested in computing the prefix sum for some $f \in R^{\mZ}$,
\[
    \Sigma_{f}(n) \coloneqq\sum_{k = 1}^{n} f(k).
\]

 If we want to compute $\Sigma_{f}(n)$, then we will search a decomposition
$h = g*f$.

We start by giving two important lemmas.
\begin{lemma}\label{lem: division lemma}
    Let $k_1,k_2,k_3 \in \mZ$, then
    \[
        \left\lfloor \frac{\left\lfloor \frac{k_1}{k_2} \right\rfloor }{k_3} \right\rfloor = 
        \left\lfloor \frac{k_1}{k_2 \cdot k_3} \right\rfloor .
    \]
\end{lemma}
\begin{proof}
    By the division algorithm for positive integers, there exists unique $q_1,r_1 \in \NN$ with
    $r_1 < k_2$ such that $k_1 = k_2 q_1 + r_1$.
    $q_1$ coincides with $\left\lfloor \frac{k_1}{k_2} \right\rfloor$.
    
    Again by the division algorithm for positive integers, there exists unique $q_2,r_2 \in \NN$ with
    $r_2 < k_3$ such that $q_1 = k_3 q_2 + r_2$.
    Therefore, 
    $q_2 = \left\lfloor \frac{q_1}{k_3} \right\rfloor = \left\lfloor \frac{\left\lfloor \frac{k_1}{k_2} \right\rfloor}{k_3}\right\rfloor$.
    
    Substituting $q_1 = k_3 q_2 + r_2$ in the first division yields that 
    \[
    k_1 = (k_2 k_3) q_2 + (k_2 r_2 + r_1).
    \]

    Next we prove that the previous equation is the integer division of $k_1$ by $k_2 k_3$. Since integer division is unique, we only need to check that $q_2, (k_2 r_2 + r_1) \in \NN$ and
    $k_2 r_2 + r_1 < k_2 k_3$. The first condition is obviously true and for the second one
    \[
    k_2 r_2 + r_1 < k_2r_2 + k_2 = k_2 (r_2 + 1) \leq k_2 k_3.
    \]

    By the uniqueness of integer division, 
    \[
    \left\lfloor \frac{\left\lfloor \frac{k_1}{k_2} \right\rfloor}{k_3}\right\rfloor = q_2 = \left\lfloor \frac{k_1}{k_2 k_3} \right\rfloor.
    \]
\end{proof}

\begin{lemma}\label{lemma: harmonic}
    Let $n \in \mZ$. Then
        \[
        \Card\left(\left\{ \left\lfloor \frac{n}{d}\right\rfloor \, \middle| \,  1 \leq d \leq n\right\}\right) \leq 2 \lfloor \sqrt{n} \rfloor.
        \]
\end{lemma}
\begin{proof}
         \[
        \Card\left(\left\{ \left\lfloor \frac{n}{d}\right\rfloor \, \middle| \,  1 \leq d \leq  \lfloor \sqrt{n} \rfloor\right\}\right) \leq \lfloor \sqrt{n} \rfloor.
        \]

        For every $d \in \{ \lfloor \sqrt{n} \rfloor +1,\dots, n\}$,
        $ 1 \leq \left\lfloor \frac{n}{d}\right\rfloor \leq \lfloor \sqrt{n} \rfloor$ since $0 = \left\lfloor \frac{n}{d \cdot(\lfloor \sqrt{n} \rfloor +1)}\right\rfloor
        =\left\lfloor \frac{\left\lfloor \frac{n}{d}\right\rfloor}{\lfloor \sqrt{n} \rfloor +1} \right\rfloor $. Therefore, 
                \[
                \begin{aligned}
        \Card &\left(\left\{ \left\lfloor \frac{n}{d}\right\rfloor \, \middle| \,   1 \leq d \leq n \right\}\right) = \\
           &\Card\left(\left\{ \left\lfloor \frac{n}{d}\right\rfloor \, \middle| \,  1 \leq d \leq \lfloor \sqrt{n} \rfloor\right\}\right) +
              \Card\left(\left\{ \left\lfloor \frac{n}{d}\right\rfloor \, \middle| \,  \lfloor \sqrt{n} \rfloor+1 \leq d \leq n\right\}\right)
        \leq 2 \lfloor \sqrt{n} \rfloor.
                        \end{aligned}
        \]

\end{proof}

Now we begin with the optimization.
We suppose that $f \in R^{\mZ}$ can be decomposed as $h = f*g$, so more sophisticated techniques can be used to exploit the 
structure of \(f\). Then it is satisfied that
\[
\begin{aligned}
    \Sigma_{h}(n) &=\sum_{k = 1}^{n} h(k) =
   \sum_{k = 1}^{n} \sum_{d \mid k} g(d) \cdot f\left( \frac{k}{d}\right) = \\ &
   \sum_{d = 1}^{n} 
    \sum_{r = 1}^{\left\lfloor \frac{n}{d}  \right\rfloor} g(d) \cdot f(r) =
   \sum_{d = 1}^{n} g(d) \cdot
   \left( \sum_{r = 1}^{\left\lfloor \frac{n}{d}  \right\rfloor} f(r)\right) = \sum_{d = 1}^{n} g(d) \cdot \Sigma_{f}\left( \left\lfloor \frac{n}{d}  \right\rfloor \right).
\end{aligned}
\]

We can isolate the term for $d = 1$, which gives us 
\[
\begin{aligned}
    \Sigma_{h}(n) &= g(1) \cdot \Sigma_{f}\left( 
   n \right) +\sum_{d = 2}^{n} g(d) \cdot \Sigma_{f}\left( \left\lfloor \frac{n}{d}  \right\rfloor \right).
\end{aligned}
\]

If $g(1) \neq 1$ and we can solve for $\Sigma_{f}$, which yields
\[
    \Sigma_{f}(n) = \frac{\Sigma_{h}(n) - \sum_{d = 2}^{n} g(d) \cdot \Sigma_{f}\left( \left\lfloor \frac{n}{d}  \right\rfloor \right)}{g(1)}.
\]

Let 
\begin{gather*}
      A^{-}(n) \coloneqq \left\{ \left\lfloor \frac{n}{d}\right\rfloor \, \middle| \,  1 \leq d \leq  \lfloor \sqrt{n} \rfloor\right\}, \\
         A^{+}(n) \coloneqq \left\{ \left\lfloor \frac{n}{d}\right\rfloor \, \middle| \,   \lfloor \sqrt{n} \rfloor+ 1 \leq d \leq n\right\},\\
         A(n) \coloneqq A^{-}(n) \sqcup A^{+}(n).
\end{gather*}

As we have seen in Lemma \ref{lemma: harmonic}, the elements of
$A^{+}(n)$ are bounded above by $\lfloor \sqrt{n} \rfloor$ while the elements of $A^{-}(n)$ are bounded bellow by $\lfloor \sqrt{n} \rfloor$, since the sequence $\left\lfloor \frac{n}{1} \right\rfloor, \left\lfloor \frac{n}{3} \right\rfloor,\dots, \left\lfloor \frac{n}{n} \right\rfloor$ is non-increasing and
$\lfloor \sqrt{n} \rfloor^{2} \leq n$.
There are a lot more indices in $A^{+}(n)$ than in $A^{-}(n)$, but
in  $A^{+}(n)$ several indices collapse to the same value.

Since the sequence $\left\lfloor \frac{n}{1} \right\rfloor, \left\lfloor \frac{n}{3} \right\rfloor,\dots, \left\lfloor \frac{n}{n} \right\rfloor$ is non-increasing, we can compute an interval partition of $\{1,\dots,n\}$ such that each interval represents
a value of the sequence. Suppose we have $d \in \{1,\dots,n\}$, 
how can we compute the interval that contains $d$?

Let $\left \lfloor \frac{n}{d} \right\rfloor = k$, then for every element $x$ of the same interval as $d$ is satisfied that
\[
 k \leq \frac{n}{x} < k+1.
\]

Solving for $x$ we get that $\left \lfloor \frac{n}{x} \right\rfloor = k$ if and only if
\[
\frac{n}{k+1} < x \leq \frac{n}{k}.
\]
That last condition can be written as
$x \in \left\{ \left\lfloor \frac{n}{k+1} \right\rfloor + 1, \dots,
 \left\lfloor \frac{n}{k} \right\rfloor \right\}$, so the interval of
 $d$ is 
 \[\left\{ \left\lfloor \frac{n}{\left \lfloor \frac{n}{d} \right\rfloor+1} \right\rfloor + 1, \dots,
 \left\lfloor \frac{n}{\left \lfloor \frac{n}{d} \right\rfloor} \right\rfloor \right\}.\]

 This characterization and the Lemma \ref{lemma: harmonic} gives us an algorithm in $\cO(\sqrt{n})$ that computes iteratively the desired interval partition. Let $\overline{A}$ be such partition. The formula for the prefix sum becomes
 \[
    \Sigma_{f}(n) = \frac{\Sigma_{h}(n) - \sum_{(a,b) \in \overline{A}}  \left( \Sigma_{g}(b) - \Sigma_{g}(a-1)\right)\cdot \Sigma_{f}\left( \left\lfloor \frac{n}{a}  \right\rfloor \right)}{g(1)}.
\]

At first it may seem that this formula is recursive but
Lemma \ref{lem: division lemma} assures us we only need to compute the values that belongs to $A(n)$ in all the recursive calls of the function. Therefore, this can be put as a triangular linear system of
$\cO(\sqrt{n})$ variables and equations that can be solved via substitution. It is important to notice that the equation with value
$\left\lfloor \frac{n}{d} \right\rfloor$ associated involves at most
$ 2 \cdot \sqrt{\left\lfloor \frac{n}{d} \right\rfloor}$ terms by Lemma \ref{lemma: harmonic}.
We suppose that we can compute $\Sigma_{g}(k)$  in $\cO(1)$ and $\Sigma_{h}(k)$  in $\cO(\sqrt{k})$.

In conclusion, the cost for solving the variables belonging to 
$A^{+}(n)$ is in $\cO\left(\sum_{i = 1}^{\lfloor \sqrt{n} \rfloor}\sqrt{i}\right)$, where it is included the cost for computing $\Sigma_{h}$.

In order to better understand the cost, we prove the following lemma
\begin{lemma}
For every $k \in \mZ$,
    \[
    \sum_{i = 1}^{k} \sqrt{i} \leq k^{\frac{3}{2}}.
    \]
\end{lemma}
\begin{proof}
    
We will show it by induction on $k$.
It is obviously true for $k = 1$ and if it is satisfied by $k$,
then 
\[
    \sum_{i = 1}^{k+1} \sqrt{i} =
    \sum_{i = 1}^{k} \sqrt{i} + \sqrt{k+1} \leq 
    \sum_{i = 1}^{k} \sqrt{i} + \sqrt{3k} \leq
    \sqrt{k^{3}} + \sqrt{3k} \leq \sqrt{k^{3} + 3k^{2} + 3k +1} = (k+1)^{\frac{3}{2}}.
\]

\end{proof}

Therefore,
\[
    \sum_{i = 1}^{\lfloor \sqrt{n} \rfloor} \sqrt{i} \leq 
    \lfloor \sqrt{n} \rfloor^{\frac{3}{2}} \leq 
     \left(\sqrt{n}\right)^{\frac{3}{2}} = n^{\frac{3}{4}}.
\]

We deduce that the cost for solving the variables belonging to 
$A^{+}(n)$ is in $\cO(n^{\frac{3}{4}})$.

Now, for solving the variables of $A^{-}(n)$, the cost is given by
$\cO\left( \sum_{i = 1}^{\lfloor \sqrt{n} \rfloor} \sqrt{\left\lfloor \frac{n}{i} \right\rfloor} \right) = 
\cO\left( \sum_{i = 1}^{\lfloor \sqrt{n} \rfloor} \frac{\sqrt{n} }{\sqrt{i}} \right) = \cO\left( \sum_{i = 1}^{\lfloor \sqrt{n} \rfloor} \frac{\lfloor \sqrt{n} \rfloor}{\sqrt{i}} \right)$, where it is included the cost for computing $\Sigma_{h}$.

As before, in order to better understand the cost, we prove 
a lemma.

\begin{lemma}
    For every $k \in\mZ $,
    \[
    \sum_{i = 1}^{k} \frac{k}{\sqrt{i}} \leq 8\sqrt{k^{3}}.
    \]
\end{lemma}
\begin{proof}
    
Again, we show it by induction on $k$. 
It is obviously true for $k = 1$ and if it is satisfied by $k$,
then 
\[
\begin{aligned}
    \sum_{i = 1}^{k+1} \frac{k+1}{\sqrt{i}} &=
    \sum_{i = 1}^{k} \frac{k+1}{\sqrt{i}} +\frac{k}{\sqrt{k+1}} \leq 
    \frac{k+1}{k} \cdot \sum_{i = 1}^{k}  \frac{k}{\sqrt{i}} + \frac{k}{\sqrt{k+1}} \leq\\
    & \frac{8(k+1)}{k}\sqrt{k^{3}} + \frac{k}{\sqrt{k+1}} =
    8(k+1)\sqrt{k} + \frac{k \sqrt{k+1}}{k+1}.
\end{aligned}
\]

The square of this last term satisifies that 
\[
    \begin{aligned}
            \left(8(k+1)\sqrt{k} + \frac{k \sqrt{k+1}}{k+1}\right)^{2} &=
            64(k^{3} + 2k^{2} + k)  + \frac{k^{2}}{k+1} + 16k\sqrt{k (k+1)} \leq\\
            &64(k^{3} + 2k^{2} + k)  + k + 16 k(k+1) =
            64k^{3} + 144k^{2} +18k <\\
            & 64k^{3} + 192k^{2} + 192k + 64 = \left( 8 \sqrt{k+1} \right)^{2}.
    \end{aligned}
\]

Therefore, we have proved that for $k \in \mZ$
\[
 \sum_{i = 1}^{k} \frac{k}{\sqrt{i}} \leq 8 \sqrt{k^{3}}.
\]
\end{proof}


Substituting $k = \lfloor \sqrt{n} \rfloor$ in the last equation
proves that solving the variables in $A^{-}(n)$ can be done in
$\cO\left( \sum_{i = 1}^{\lfloor \sqrt{n} \rfloor} \frac{\lfloor \sqrt{n} \rfloor}{\sqrt{i}} \right) \subseteq \cO\left(n^{\frac{3}{4}} \right)$

In conclusion, the whole algorithm runs in $\cO(n^{\frac{3}{4}})$.



Nevertheless, if $f$ is multiplicative we can precompute the values of $\Sigma_{f}$ in the range $\{1,\dots,m\}$ in $\cO(m)$ with a sieve.
Since we already have a cost in $\Omega(\sqrt{n})$ by iterating over the interval partition, we can suppose that $\lfloor \sqrt{n} \rfloor < m$. Therefore, our total cost would be in $\cO(m)$ for precomputing $\Sigma_{f}$ with a sieve and then
$\cO\left( \sum_{i = 1}^{\left\lfloor \frac{n}{m} \right\rfloor} \sqrt{\left\lfloor \frac{n}{i} \right\rfloor} \right) = \cO\left( \sqrt{m} \cdot \sum_{i = 1}^{\left\lfloor \frac{n}{m} \right\rfloor} \sqrt{\frac{ \left\lfloor \frac{n}{m} \right\rfloor}{i}
} \right) = \cO\left( \frac{\sqrt{m^{2}}}{\sqrt{n}} \cdot \sum_{i = 1}^{\left\lfloor \frac{n}{m} \right\rfloor} \frac{ \left\lfloor \frac{n}{m} \right\rfloor}{\sqrt{i}}
 \right) \subseteq
\cO\left( \frac{n}{\sqrt{m}} \right)$ for solving the remaining values of $\Sigma_{f}$ in $A^{-}(n)$.
Taking $\alpha \in [0,1]$ and $m = n^{\alpha}$ yields the total cost of
$\cO\left(\max\left\{ n^{\alpha}, n^{1 -\frac{\alpha}{2}}\right\}\right)$, which is minimized with
$\alpha = \frac{2}{3}$. 

The conclusion is that computing the prefix sum of such a multiplicative function can be done in $\cO\left(n^{\frac{2}{3}}\right)$ precomputing with a sieve $\Sigma_{f}$ until $\left\lfloor n^{\frac{2}{3}} \right\rfloor$ and then applying the previous algorithm. This method uses $\cO\left(n^{\frac{2}{3}}\right)$ extra space and can be further optimized with a segmented sieve reaching $\cO\left(\sqrt{n}\right)$ in extra space.

  



\newpage
\subsection*{Problems}

\newpage