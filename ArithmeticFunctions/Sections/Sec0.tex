\section{Introduction}

We fix a domain $(R,+,\cdot)$. Our interest lies in the space of functions 
\[ R^{\mZ} \coloneqq \{ f \mid f: \mZ \rightarrow R  \}.\]

\begin{definition}
We say that $f \in R^{\mZ}$ 
is \textbf{completely additive} if given any $n,m \in \mZ$, then f satisfies that $f(n m) = f(n) + f(m)$.
\end{definition}

\begin{definition}
We say that $f \in R^{\mZ}$ 
is \textbf{completely multiplicative} if given any $n,m \in \mZ$, then f satisfies that $f(n m) = f(n)f(m)$.
This means that $f$ is a multiplicative homomorphism of monoids.
\end{definition}

\begin{definition}
Given $f \in R^{\mZ}$, we say that $f$ is \textbf{additive} if given any $n,m \in \mZ$ such that
$\gcd(n,m) = 1$, then f satisfies that $f(n m) = f(n) + f(m)$.

It is clear that every completely additive function is additive.

We define 
\[ \Add(R) \coloneqq \{ f \in R^{\mZ} 
\mid   f \text{ is additive}\}.\]
\end{definition}

\begin{definition}
Given $f \in R^{\mZ}$, we say that $f$ is \textbf{multiplicative} if $f$ is not identically zero and given any $n,m \in \mZ$ such that
$\gcd(n,m) = 1$,  f satisfies that $f(n m) = f(n)f(m)$.

It is clear that every completely multiplicative function not identically zero is multiplicative.

We define 
\[ \Mult(R) \coloneqq \{ f \in R^{\mZ} 
\mid   f \text{ is multiplicative}\}.\]
\end{definition}

\begin{definition}
    A function $f \in R^{\mZ}$ is called \textbf{arithmetic} if $f$ is additive or multiplicative.
\end{definition}


\begin{remark}\label{rem: mul basic}
    \begin{enumerate}
           \item[(i)] Every additive function $f \in \Add(R)$  satisfies that $f(1) = 0_{R}$. 
        
        \begin{proof}
        
        We have that $f(1) = f(1 \cdot 1) = f(1) + f(1)$, from which it follows that $f(1) = 0_{R}$. 
    
        \end{proof}
        \item[(ii)] Every multiplicative function $f$  satisfies that $f(1) = 1_{R}$. 
        
        \begin{proof}
        Since $f$ is not identically zero,
         there exists $n \in \mZ$ such that $f(n) \neq 0_{R}$.

        
        Therefore, $f(n) \cdot 1_{R} = f(n) = f(n \cdot 1) = f(n) \cdot f(1)$, from which it follows that $f(1) = 1_{R}$ by the cancellation property of a domain. 
    
        \end{proof}

        \item[(iii)]     On the one hand,
    since the set of prime numbers forms a multiplicative basis of the monoid $\mZ$, a completely additive/multiplicative function is defined by its values at each prime. Moreover, any assignment of values to the primes extends uniquely to a completely additive/multiplicative function.

    On the other hand,  arithmetic functions are
    completely determined by its values on prime powers and as in the previous case, every possible evaluation on prime powers gives rise to a arithmetic function.
    \end{enumerate}
\end{remark}

We list numerous important examples of arithmetic functions.
\begin{example}
\begin{enumerate}
    \item[(i)] The  natural logarithm, 
    $\log: \mZ \rightarrow \RR$ is an (completely) additive function.
    \item[(ii)] The function $\Omega:\mZ \rightarrow \NN$ defined as the number of  prime factors with multiplicities is completely additive. If $p_1, \dots,p_r$ are distinct primes,
    then $\Omega\left(p_1^{\alpha_1}\cdot \dots \cdot p_r^{\alpha_r}\right) = \alpha_1 + \dots + \alpha_r $.

       \item[(iii)] The function $\omega:\mZ \rightarrow \NN$ defined as the number of distinct prime factors  is completely additive. If $p_1, \dots,p_r$ are distinct primes,
    then $\omega\left(p_1^{\alpha_1}\cdot \dots \cdot p_r^{\alpha_r}\right) = r $.

  
  
    \item[(iv)] Let $\tau$ denote the function $\tau: \mZ \rightarrow \mZ$ that counts the number of positive divisors of a number. If we write $n = p_{1}^{k_1} \cdot \dots \cdot p_{r}^{k_{r}}$ where $p_1 ,\dots, p_r$ are distinct primes, 
    the positive divisors of $n$ correspond bijectively to elements of the set $A \coloneqq\{0, \dots,k_1 \} \times \dots \times \{0,\dots,k_r\}$, where the $i$-th component represents the exponent of the $i$-th prime in the divisor.
    
    Therefore, $\tau(p_{1}^{k_1} \cdot \dots \cdot p_{r}^{k_{r}}) = (k_1 + 1) \cdot \dots \cdot (k_r +1)$. 
    It is obvious that this function is multiplicative.

        \item[(v)] Let $\sigma$ denote the function $\sigma: \mZ \rightarrow \mZ$ that counts the sum of the positive divisors of a number. For a prime power $p^{k}$,
        it is easy to see that 
        $\sigma(p^{k}) = 1+p+\dots+p^{k} = \frac{p^{k+1}-1}{p-1}$. If we write $n = p_{1}^{k_1} \cdot \dots \cdot p_{r}^{k_{r}}$ where $p_1 ,\dots, p_r$ are distinct primes, then
        \[
        \sigma( p_{1}^{k_1})  \dots  \sigma(p_{r}^{k_{r}}) = 
        (1+p_1 +\dots p_1^{k_1}) \dots (1+p_r +\dots p_r^{k_r}) =
        \sum_{(a_1,\dots,a_r) \in A} p_{1}^{a_1} \cdot \dots \cdot p_{r}^{a_{r}} = \sigma(p_{1}^{k_1} \cdot \dots \cdot p_{r}^{k_{r}}).
        \]
    
    It follows that $\sigma$ is multiplicative. 

    \item[(vi)]
  Let $\mu : \mZ \to \{-1,0,1\}$ denote the \textbf{Möbius} function.
It is defined by
\[
\mu(1) = 1, \quad \mu(p) = -1 \text{ for each prime } p, \quad \mu(p^k) = 0 \text{ for } k \ge 2,
\]
and extended multiplicatively to all positive integers, i.e.,
\[
\mu\Big(p_1^{k_1} \cdots p_r^{k_r}\Big) = \mu(p_1^{k_1}) \cdots \mu(p_r^{k_r}).
\]

Equivalently, for any positive integer $n$:
\[
\mu(n) =
\begin{cases}
0, & \text{if $n$ is not square-free},\\[2mm]
(-1)^r, & \text{if $n$ is the product of $r$ distinct primes}.
\end{cases}
\]

Thus, $\mu$ is a multiplicative function that vanishes on non-square-free numbers and alternates in sign depending on the parity of the number of prime factors for square-free numbers.


\item[(vii)]
Let $\varphi : \mZ \to \mZ$ denote the Euler totient function, defined by
\[
\varphi(n) \coloneqq \Card(\{ d \in \{1,\dots,n\} \mid \gcd(d,n)=1\}).
\]

We prove that $\varphi$ is multiplicative. Let $n,m\in\mZ$ with $\gcd(n,m)=1$.
For $x\in\{n,m,nm\}$ set
\[
B_x \coloneqq \{ d \in \{1,\dots,x\} \mid \gcd(d,x)=1\}.
\]

Define a map $f: B_n\times B_m \to B_{nm}$ by
\[
f(a,b) \equiv bn + am \pmod{nm},
\]
where we take the representative in $\{1,\dots,nm\}$ of the residue class of $bn+am$.

Since $\gcd(n,m)=1$ we have
\[
\gcd(f(a,b),n)=\gcd(bn+am,n)=\gcd(am,n)=\gcd(a,n)=1,
\]
and similarly $\gcd(f(a,b),m)=\gcd(b,m)=1$. Hence $f(a,b)\in B_{nm}$, which implies that $f$ is well-defined.

Next we prove the injectivity of $f$. If $f(a_1,b_1)=f(a_2,b_2)$ then
\[
(b_1-b_2)n \equiv (a_2-a_1)m \pmod{nm}.
\]
As $\gcd(n,m)=1$, it follows that $n\mid(a_2-a_1)$ and $m\mid(b_1-b_2)$.
But $a_1,a_2\in\{1,\dots,n\}$ and $b_1,b_2\in\{1,\dots,m\}$, so
$n\mid(a_2-a_1)$ (resp. $m\mid(b_1-b_2)$) implies $a_1=a_2$ (resp. $b_1=b_2$).
Thus $f$ is injective.

We finnish by proving the surjectivity of $f$. Let $s\in B_{nm}$, so $\gcd(s,nm) = 1 =\gcd(s,n) = \gcd(s,m)$. By Bézout's identity
there exist integers $u,v$ with $un+vm=1$. Then
\[
s = s(un+vm) = (su)n + (sv)m.
\]
Put $a \equiv sv \pmod{n}$ and $b \equiv su \pmod{m}$ (take the representatives
in $\{1,\dots,n\}$ and $\{1,\dots,m\}$ respectively). Since $\gcd(s,n)=\gcd(v,n)=1$
and $\gcd(s,m)=\gcd(u,m)=1$, we have $(a,b)\in B_n\times B_m$ and
$f(a,b)\equiv bn+am \equiv (su)n+(sv)m \equiv s \pmod{nm}$. Hence $f(a,b)=s$ and
$f$ is surjective.

Therefore $f$ is a bijection and
\[
\varphi(nm)=\Card( B_{nm})=\Card( B_n\times B_m)
=\Card( B_n)\cdot\Card( B_m)=\varphi(n)\varphi(m),
\]
so $\varphi$ is multiplicative.


\item[(viii)] Let $\varepsilon : \mZ \to \{0,1\}$ be defined by
\[
\varepsilon(n) =
\begin{cases}
1, & \text{if } n = 1,\\[1mm]
0, & \text{otherwise}.
\end{cases}
\]

$\varepsilon$ is a multiplicative function.

\item[(ix)] For $r \in \RR$, we define $\id_{r}: \mZ \rightarrow \RR^{+}$ as
$\id_{r}(x) = x^{r}$. This function is (completely) multiplicative for every choice of $r$. When $r \in \NN$, we consider $\id_{r}: \mZ \rightarrow \mZ$ as a monoid endomorphism. 
We denote $\id_{0} \coloneqq I$ and $\id_{1} \coloneqq \id$.
$\id$ is also additive.
\end{enumerate}
\end{example}

Next we study several function operations over arithmetic functions.
\begin{proposition}
    Let $f,g \in \Add(R)$. Then
    \begin{enumerate}
        \item[(i)] $f + g \in \Add(R)$.
        \item[(ii)] For every $\alpha \in R$, $\alpha f \in \Add(R)$.

    \end{enumerate}
\end{proposition}
\begin{proof}
\begin{enumerate}
    \item[(i)]
    $f + g$ is additive since
for $n,m \in \mZ$ such that $\gcd(n,m) = 1$,
\[
\begin{aligned}
    (f + g) (nm)& = f(nm) + g(nm) = (f(n) + f(m)) + (g(n) + g(m)) =\\
&(f(n) + g(n)) + (f(m) + g(m)) =
(f + g) (n) + (f + g) (m).
\end{aligned}
\]

    \item[(ii)]
    $\alpha f$ is additive since
for $n,m \in \mZ$ such that $\gcd(n,m) = 1$,
\[
    (\alpha f) (nm) = \alpha \cdot f(nm) = \alpha \cdot (f(n) + f(m))  =
\alpha \cdot f(n) +  \alpha \cdot f(m) =
(\alpha f) (n) + (\alpha f) (m).
\]

\end{enumerate}
\end{proof}

\begin{proposition}
    If $f$ is completely additive and $g$ is multiplicative, 
    then $f \circ g \in \Add(R)$.
\end{proposition}
\begin{proof}
$f \circ g$ is additive since
    for $n,m \in \mZ$ such that $\gcd(n,m) = 1$,
\[
(f \circ g) (nm) = f(g(nm)) = f(g(n)g(m)) = f(g(n)) + f(g(m)) =
(f \circ g) (n) + (f \circ g) (m).
\]
\end{proof}


\begin{proposition}
    Let $f,g \in \Mult(R)$. Then
    \begin{enumerate}
        \item[(i)] $f \cdot g \in \Mult(R)$.
        \item[(ii)] If $g(n) \in R^{\times}$ for all $n \in \mZ$, then $\frac{f}{g} \in \Mult(R)$.

        \item[(iii)] $f+g \notin \Mult(R)$.
        \item[(iv)] For every $\alpha \in R \setminus\{1_R\}$, $\alpha f \not\in \Mult(R)$.
    \end{enumerate}
\end{proposition}
\begin{proof}
\begin{enumerate}
    \item[(i)]
    $f \cdot g$ is multiplicative since
for $n,m \in \mZ$ such that $\gcd(n,m) = 1$,
\[
(f\cdot g) (nm) = f(nm) \cdot g(nm) = f(n) f(m) g(n) g(m) =
f(n) g(n) f(m) g(m) =
(f\cdot g) (n) (f\cdot g) (m).
\]

    \item[(ii)]
    $\frac{f}{g}$ is multiplicative since
for $n,m \in \mZ$ such that $\gcd(n,m) = 1$,
\[
\left(\frac{f}{g}\right) (nm) = \frac{f(nm)}{g(nm)} = 
\frac{f(n) f(m)}{ g(n) g(m)} =
\frac{f(n)}{g(n)} \frac{ f(m)}{g(m)} =
\left(\frac{f}{g}\right) (n) \left(\frac{f}{g}\right) (m).
\]

   \item[(iii)] 
 By Remark~\ref{rem: mul basic}, every multiplicative function satisfies $f(1)=1_R$.  
    But
    \[
    (f+g)(1)=f(1)+g(1)=1_R+1_R = 2_R \neq 1_R,
    \]
    so  $f+g$ cannot be multiplicative. Hence  
    \(f+g \notin \Mult(R)\).
    
          \item[(iv)]
    Again, every multiplicative function must take value $1_R$ at $1$. But
    \[
    (\alpha f)(1) = \alpha f(1)= \alpha \neq 1_R,
    \]
    since $\alpha \in R\setminus\{1_R\}$.  Hence  
    \(\alpha f\notin \Mult(R)\).
\end{enumerate}
\end{proof}

\begin{proposition}
    If $f$ is completely multiplicative and $g$ is multiplicative, 
    then $f \circ g \in \Mult(R)$.
\end{proposition}
\begin{proof}
$f \circ g$ is multiplicative since
    for $n,m \in \mZ$ such that $\gcd(n,m) = 1$,
\[
(f \circ g) (nm) = f(g(nm)) = f(g(n)g(m)) = f(g(n)) \cdot f(g(m)) =
(f \circ g) (n) \cdot (f \circ g) (m).
\]
\end{proof}

\newpage
\subsection*{Problems}

\begin{exercise}
    For $\alpha \in \RR$, let 
    $\sigma_{\alpha}: \mZ \rightarrow \RR^{+}$ be defined as
    \[
    \sigma_{\alpha}(n) \coloneqq \sum_{d \mid n} d^{\alpha}.
    \]
    If $\alpha \in \NN$, we can consider $\sigma_{\alpha}$ as an endomorphism. 

    Prove that for every choice of $\alpha \in \RR$, $\sigma_{\alpha}$ is multiplicative.
    Find the multiplicative functions that were presented in the examples that belong to the familiy
    $\{ \sigma_{\alpha}\}_{\alpha \in \RR}$.
\end{exercise}

\begin{exercise}
    Prove that for every prime power $p^{k}$ with $k \in \mZ$,
    it is satisfied that $\varphi(p^{k}) = p^{k-1}(p-1)$.
    Using the fact that $\varphi$ is multiplicative, conclude the general formula for $\varphi$, with the input given as a product of distinct primes.
\end{exercise}


\begin{exercise}
    Classify all the arithmetic functions with $R = \ZZ_{2}$.
\end{exercise}

\newpage