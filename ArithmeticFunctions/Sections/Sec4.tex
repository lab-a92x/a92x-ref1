\subsection{Dirichlet hyperbola method}
We are interested in computing the prefix sum for some $f \in R^{\mZ}$,
\[
    \Sigma_{f}(n) \coloneqq\sum_{k = 1}^{n} f(k).
\]

As we have seen, computing prefix quickly can solve efficiently several problems. 
A classical approach for computing prefix sums of 
$f \in  R^{\mZ}$ 
is based on the \textbf{Dirichlet hyperbola method}.


Let $f \in  R^{\mZ}$ and $n$ be fixed. Suppose we are given
$f = g*h$ for some $g,h \in R^{\mZ}$.
Then we can write
\[
\sum_{k = 1}^{n} f(k) = \sum_{k = 1}^{n} \sum_{xy= k} g(x) \cdot h(y).
\]

At first it may seem that our situation has worsened but we can view
the two sumatoriums as the integer points below the hyperbola
$\{ (x,y) \in \RR^{2} \mid x\cdot y = n\}$, that is,
the region where we have to sum the values of $g(x) \cdot h(y)$ is 
\[
Z \coloneqq \{ (x,y) \in \mZ \times \mZ \mid x \cdot y \leq n\}\]


% Define N before the tikzpicture
\def\N{15}

\begin{tikzpicture}[scale=0.4]

% --- Draw integer grid ---
\foreach \x in {1,...,\N}{
    \draw[gray!30] (\x,0) -- (\x,\N);
}
\foreach \y in {1,...,\N}{
    \draw[gray!30] (0,\y) -- (\N,\y);
}

% --- Axes ---
\draw[->] (0,0) -- (\N+1,0) node[right] {$x$};
\draw[->] (0,0) -- (0,\N+1) node[above] {$y$};

% --- Hyperbola y = N/x ---
\draw[thick,domain=1:\N,smooth,variable=\x,black]
    plot ({\x},{\N/\x}) node[right] {$xy=n$};

% --- Highlight integer points under the hyperbola ---
\foreach \x in {1,...,\N}{
    \foreach \y in {1,...,\N}{
    \pgfmathparse{\x*\y <= \N ? 1 : 0}  % evaluate the condition
        \ifnum\pgfmathresult=1
            \fill[blue!50] (\x,\y) circle (3pt);
        \fi
    }
}

% --- Title / comment below the diagram ---
\node[above,font=\small] at (\N/2,\N+2)
    {For $n = 15$, $Z$ corresponds with the blue dots of the following diagram:};


\end{tikzpicture}


Now suppose we are given $a,b \in \RR^{2}$ such that $a\cdot b = n$.
We will use $a,b$ to divide $Z$.
This division will not be disjoint, so we must take care of the intersection in order to use the inclusion-exclusion principle.
\begin{gather*}
 S_{a} := \{(x,y) \in \RR^{2} \mid xy \leq n,\, x \leq a  \},\\
 S_{b} := \{(x,y) \in \RR^{2} \mid xy \leq n, \, y \leq  b  \},\\
 Z_{a} \coloneqq S_{a} \cap Z =
 \{(x,y) \in Z \mid x \leq \lfloor a \rfloor \},\\
 Z_{b} \coloneqq S_{a} \cap Z_{b} = \{(x,y) \in Z \mid y \leq \lfloor b \rfloor \},\\
    Z_{a} \cap Z_{b} = \{(x,y) \in Z \mid x \leq \lfloor a \rfloor,\, y \leq \lfloor b \rfloor \}.
\end{gather*}

\def\N{10}
\pgfmathsetmacro{\a}{sqrt(\N)+1}
\pgfmathsetmacro{\b}{\N/\a}

\begin{tikzpicture}[scale=0.7]

% --- Grid ---
\foreach \x in {1,...,\N} \draw[gray!30] (\x,0) -- (\x,\N);
\foreach \y in {1,...,\N} \draw[gray!30] (0,\y) -- (\N,\y);

% --- Axes ---
\draw[->] (0,0) -- (\N+2,0) node[right] {$x$};
\draw[->] (0,0) -- (0,\N+2) node[above] {$y$};

% --- Hyperbola ---
\draw[thick,domain=1:\N,smooth,variable=\x] plot ({\x},{\N/\x}) node[right] {$xy=n$};

% --- S_a region under hyperbola ---
\fill[red!30,opacity=0.4] 
    (1,0)
    -- plot[domain=1:\a,smooth,variable=\x] (\x,{\N/\x})
    -- (\a,0)
    -- cycle;
\fill[red!30,opacity=0.4] 
(0,0)
-- plot[domain=0:1,smooth,variable=\x] (\x,{\N})
-- (1,0)
-- cycle;

% --- S_b region under hyperbola ---
\fill[blue!30,opacity=0.4]
    (0,1)
    -- plot[domain=1:\b,smooth,variable=\y] ({\N/\y},\y)
    -- (0,\b)
    -- cycle;
\fill[blue!30,opacity=0.4]
(0,0)
-- plot[domain=0:1,smooth,variable=\y] ({\N},\y)
-- (0,1)
-- cycle;

% --- Point (a,b) ---
\fill[black] (\a,\b) circle (3pt) node[above right] {$(a,b)$};

% Compute integer versions of a and b
\pgfmathtruncatemacro{\afloor}{floor(\a)}
\pgfmathtruncatemacro{\bfloor}{floor(\b)}

% --- Integer points colored by region 
\foreach \x in {1,...,\N}{
    \foreach \y in {1,...,\N}{
        % Check if the point is under the hyperbola xy <= N
        \pgfmathparse{\x*\y <= \N ? 1 : 0}
        \ifnum\pgfmathresult=1
            % Check intersection Z_a ∩ Z_b
            \pgfmathparse{ \x <= \afloor} % 1 if  true
            \ifnum\pgfmathresult=1
                \pgfmathparse{ \y <= \bfloor} % 1 if  true
                \ifnum\pgfmathresult=1
                    \fill[purple!50!magenta] (\x,\y) circle (3pt);
                \else
                    \fill[red] (\x,\y) circle (3pt);
                \fi
            \else
                \fill[blue] (\x,\y) circle (3pt);
            \fi
        \fi
    }
}



% --- Labels ---
\node[red!70] at (\a/3,\N) {$S_a$};
\node[blue!70] at (\N,\b/2 + 0.5) {$S_b$};

% --- Comment below ---
\node[below,font=\small,align=center,text width=12cm] at (\N/2,-1)
{
For $n = 10$ and $a = \sqrt{n} + 1$, $b = \frac{n}{a}$, we represent the
region $S_a$ (red) and $S_b$ (blue) under the hyperbola. 
The integer points in $Z_{a}\setminus Z_{b}$ are red, the points in $Z_{b}\setminus Z_{a}$ are blue, and the points in $Z_{a} \cap Z_{b}$ are purple. 
};

\end{tikzpicture}

Therefore, by the inclusion-exclusion principle, we obtain that
\begin{equation*}\label{eq: Dirich hyper meth}
\begin{aligned}
    \Sigma_f(n) =\sum_{k = 1}^{n} f(k) =& 
    \sum_{k = 1}^{n} \sum_{xy = k} g(x)\cdot h(y)
    = \sum_{(x,y) \in Z} g(x)\cdot h(y) =\\ 
    &\sum_{(x,y) \in Z_{a}} g(x)\cdot h(y)
    + \sum_{(x,y) \in Z_{b}} g(x)\cdot h(y) 
    - \sum_{(x,y) \in Z_{a} \cap  Z_{b}} g(x)\cdot h(y) =\\
    & \sum_{x = 1}^{\lfloor a \rfloor} 
    \sum_{y = 1}^{\lfloor \frac{n}{x} \rfloor} g(x)\cdot h(y)+
    \sum_{y = 1}^{\lfloor b \rfloor} 
    \sum_{x = 1}^{\lfloor \frac{n}{y} \rfloor} g(x)\cdot h(y)-
    \sum_{x = 1}^{\lfloor a \rfloor} 
    \sum_{y = 1}^{\lfloor b \rfloor} g(x) \cdot h(y) =\\
    &\sum_{x = 1}^{\lfloor a \rfloor}\left( g(x)\cdot \sum_{y = 1}^{\lfloor \frac{n}{x} \rfloor}  h(y)\right) +
     \sum_{y = 1}^{\lfloor b \rfloor} \left( h(y)\cdot \sum_{x = 1}^{\lfloor \frac{n}{y} \rfloor}  g(x)\right) -
     \left(  \sum_{x = 1}^{\lfloor a\rfloor}  g(x)\right) \cdot \left( \sum_{y = 1}^{\lfloor b  \rfloor}  h(y)\right).
\end{aligned}
\end{equation*}




We will see several examples where this method provides a cost in 
$\cO(\sqrt{n})$ taking $a = b = \sqrt{n}$.
\begin{example}
\leavevmode
\begin{enumerate}
    \item[(i)]
    Since $\tau = s * s$, if we take $a=b=\sqrt{n}$ and $g=h=s$, then it follows that

\[
\begin{aligned}
    \sum_{k = 1}^{n} \tau(k) =& \sum_{x = 1}^{\lfloor \sqrt{n} \rfloor} 
    \sum_{y = 1}^{\left\lfloor \frac{n}{x} \right\rfloor} s(x)\cdot s(y)+
    \sum_{y = 1}^{\lfloor \sqrt{n} \rfloor} 
    \sum_{x = 1}^{\left\lfloor \frac{n}{y} \right\rfloor} s(x)\cdot s(y)-
    \sum_{x = 1}^{\lfloor \sqrt{n} \rfloor} 
    \sum_{y = 1}^{\lfloor \sqrt{n} \rfloor} s(x) \cdot s(y) = \\
    &\sum_{x = 1}^{\lfloor \sqrt{n} \rfloor} 
    \sum_{y = 1}^{\left\lfloor \frac{n}{x} \right\rfloor} 1+
    \sum_{y = 1}^{\lfloor \sqrt{n} \rfloor} 
    \sum_{x = 1}^{\left\lfloor \frac{n}{y} \right\rfloor} 1-
    \sum_{x = 1}^{\lfloor \sqrt{n} \rfloor} 
    \sum_{y = 1}^{\lfloor \sqrt{n} \rfloor} 1  =
    \sum_{x = 1}^{\lfloor \sqrt{n} \rfloor} 
    \left\lfloor \frac{n}{x} \right\rfloor+
    \sum_{y = 1}^{\lfloor \sqrt{n} \rfloor} 
    \left\lfloor \frac{n}{y} \right\rfloor-
    \sum_{x = 1}^{\lfloor \sqrt{n} \rfloor} 
    \lfloor \sqrt{n} \rfloor=\\
    &2 \cdot\left(\sum_{x = 1}^{\lfloor \sqrt{n} \rfloor}
       \left\lfloor \frac{n}{x} \right\rfloor \right) - \lfloor \sqrt{n} \rfloor^{2}.
\end{aligned}
\]

\item[(ii)]
    Since $\sigma = \id * s$, if we take $a=b=\sqrt{n}$, $g=\id$ and $h=s$, following the previous example we derive

\[
\begin{aligned}
    \sum_{k = 1}^{n} \sigma(k) =& \sum_{x = 1}^{\lfloor \sqrt{n} \rfloor} 
    \sum_{y = 1}^{\left\lfloor \frac{n}{x} \right\rfloor} \id(x)\cdot s(y)+
    \sum_{y = 1}^{\lfloor \sqrt{n} \rfloor} 
    \sum_{x = 1}^{\left\lfloor \frac{n}{y} \right\rfloor} \id(x)\cdot s(y)-
    \sum_{x = 1}^{\lfloor \sqrt{n} \rfloor} 
    \sum_{y = 1}^{\lfloor \sqrt{n} \rfloor} \id(x) \cdot s(y) = \\
    &\sum_{x = 1}^{\lfloor \sqrt{n} \rfloor} 
    \sum_{y = 1}^{\left\lfloor \frac{n}{x} \right\rfloor} x+
    \sum_{y = 1}^{\lfloor \sqrt{n} \rfloor} 
    \sum_{x = 1}^{\left\lfloor \frac{n}{y} \right\rfloor} x-
    \sum_{x = 1}^{\lfloor \sqrt{n} \rfloor} 
    \sum_{y = 1}^{\lfloor \sqrt{n} \rfloor} x  =\\
    &\sum_{x = 1}^{\lfloor \sqrt{n} \rfloor} 
    x \cdot \left\lfloor \frac{n}{x} \right\rfloor+
    \sum_{y = 1}^{\lfloor \sqrt{n} \rfloor} 
    \frac{\left\lfloor \frac{n}{y} \right\rfloor \cdot \left(\left\lfloor \frac{n}{y} \right\rfloor + 1\right)}{2}-
    \sum_{x = 1}^{\lfloor \sqrt{n} \rfloor} 
    \lfloor \sqrt{n} \rfloor \cdot x =\\
    &\sum_{x = 1}^{\lfloor \sqrt{n} \rfloor}
    x \cdot \left\lfloor \frac{n}{x} \right\rfloor + 
    \frac{\left\lfloor \frac{n}{x} \right\rfloor \cdot \left(\left\lfloor \frac{n}{x} \right\rfloor + 1\right)}{2} -
    \lfloor \sqrt{n} \rfloor \cdot x
    .
\end{aligned}
\]

\item[(iii)]
Let $f \in \Mult(\mZ)$ be such that $f = \id * \id$.
It is easily veryfiable that for prime powers it is satisfied that
$f(p^{k}) = (k+1)\cdot p^{k}$ and for other values we can extend multiplicatively. Again, we will give a formula for $\sum_{k=1}^{n} f(k)$, computable in $\cO(\sqrt{n})$. We take $a=b=\sqrt{n}$ and $g= h = \id$.

\[
\begin{aligned}
    \sum_{k = 1}^{n} f(k) =& \sum_{x = 1}^{\lfloor \sqrt{n} \rfloor} 
    \sum_{y = 1}^{\left\lfloor \frac{n}{x} \right\rfloor} \id(x)\cdot \id(y)+
    \sum_{y = 1}^{\lfloor \sqrt{n} \rfloor} 
    \sum_{x = 1}^{\left\lfloor \frac{n}{y} \right\rfloor} \id(x)\cdot \id(y)-
    \sum_{x = 1}^{\lfloor \sqrt{n} \rfloor} 
    \sum_{y = 1}^{\lfloor \sqrt{n} \rfloor} \id(x) \cdot \id(y) = \\
    &\sum_{x = 1}^{\lfloor \sqrt{n} \rfloor} 
    \sum_{y = 1}^{\left\lfloor \frac{n}{x} \right\rfloor} x\cdot y+
    \sum_{y = 1}^{\lfloor \sqrt{n} \rfloor} 
    \sum_{x = 1}^{\left\lfloor \frac{n}{y} \right\rfloor} x\cdot y -
    \sum_{x = 1}^{\lfloor \sqrt{n} \rfloor} 
    \sum_{y = 1}^{\lfloor \sqrt{n} \rfloor} x\cdot y  =\\
    &\sum_{x = 1}^{\lfloor \sqrt{n} \rfloor} x \cdot
    \sum_{y = 1}^{\left\lfloor \frac{n}{x} \right\rfloor} y+
    \sum_{y = 1}^{\lfloor \sqrt{n} \rfloor} y \cdot
    \sum_{x = 1}^{\left\lfloor \frac{n}{y} \right\rfloor} x -
    \sum_{x = 1}^{\lfloor \sqrt{n} \rfloor}  x\cdot 
    \sum_{y = 1}^{\lfloor \sqrt{n} \rfloor} y  =\\
      &\sum_{x = 1}^{\lfloor \sqrt{n} \rfloor} x \cdot
      \frac{\left\lfloor \frac{n}{x} \right\rfloor \cdot
      (\left\lfloor \frac{n}{x} \right\rfloor + 1)}{2}
    +
    \sum_{y = 1}^{\lfloor \sqrt{n} \rfloor} y \cdot
      \frac{\left\lfloor \frac{n}{y} \right\rfloor \cdot
      \left(\left\lfloor \frac{n}{y} \right\rfloor + 1\right)}{2} -
    \left(\frac{\lfloor \sqrt{n} \rfloor\cdot(\lfloor \sqrt{n} \rfloor + 1)}{2}\right)^{2}  =\\
    &\sum_{x = 1}^{\lfloor \sqrt{n} \rfloor} x \cdot
    \left(\left\lfloor \frac{n}{x} \right\rfloor \cdot
      \left(\left\lfloor \frac{n}{x} \right\rfloor + 1\right)\right)
      -
    \left(\frac{\lfloor \sqrt{n} \rfloor\cdot(\lfloor \sqrt{n} \rfloor + 1)}{2}\right)^{2}
    .
\end{aligned}
\] 
\end{enumerate}
    
\end{example}
Now we will continue where we left.
The equation previous to the examples can be rewritten as
\[
 \Sigma_{f}(n) = 
 \sum_{x = 1}^{\lfloor a \rfloor}\left( g(x)\cdot 
 \Sigma_{h}\left( \left\lfloor \frac{n}{x} \right\rfloor \right) \right) +
     \sum_{y = 1}^{\lfloor b \rfloor} \left( h(y)\cdot \Sigma_{g}\left(\left\lfloor \frac{n}{y} \right\rfloor \right) \right) -
     \Sigma_{g}\left(\left\lfloor a\right\rfloor\right)  \cdot \Sigma_{h}\left(\left\lfloor b\right\rfloor\right) .
\]



Suppose we can compute $\Sigma_{g}(k) \in \cO(k^{\alpha})$
and $\Sigma_{h}(k) \in \cO(k^{\beta})$ for $\alpha, \beta \in [0,1)$.
 We will select $ a = n^{\gamma}$  and
$b = n^{1-\gamma}$ for $\gamma \in [0,1]$ to minimize the cost. Using Lemma \ref{lem: sum inv cost}, the total cost of the algorithm would be in 
\[
 \cO\left(  \sum_{i = 1}^{ \left\lfloor n^{\gamma} \right\rfloor} \frac{n^{\alpha}}{i^{\alpha}} + \sum_{i = 1}^{\left\lfloor n^{1-\gamma} \right\rfloor} \frac{n^{\beta}}{i^{\beta}}  \right) = \cO\left( \max\left\{  
 	n^{\alpha + \gamma (1- \alpha)},
 	n^{\beta + (1-\gamma) (1- \beta)}\right\} \right).
\]
This cost is minimized by taking
$\alpha + \gamma (1- \alpha)
 = \beta + (1-\gamma) (1- \beta) = 
 1 - \gamma(1- \beta)$,
which yields the value 
\[
\gamma = \frac{1 - \alpha}
{2 - \alpha - \beta}.
\]

\newpage
\subsection*{Problems}

\newpage